\chapter*{General Discussion and Conclusions}
\addcontentsline{toc}{chapter}{General discussion and conclusions}
A large body of literature in neuropsychology provides varied support to the view that reading errors made by dyslexic people can result from distinct and independent causes \citep{mn73, sw77, c83, shallice1988neuropsychology, c96, shallice2000selective, friedmann2001letter, ellis2013human}. Accordingly, dyslexia has been analyzed into subtypes of dyslexia, each corresponds to a different deficit in the process of reading with different characteristics, including different error types, and different word types that cause difficulty in reading \citep[for a review, see,  ][]{ck12}. These neuropsychological studies have gained further support by research in neuroscience, looking into brain function during reading \citep{fiebach2002fmri, joubert2004neural, levy2009testing}, and were accompanied by computational simulations of the suggested underlying sub-processes \citep{coltheart2001drc, perry2007nested}. This thread of research has been continuously striving to identify new subtypes of dyslexia in order to improve our understanding of the process of reading and to advance therapeutic methods. In particular, recently, a new family of subtypes of dyslexia was identified, which was suggested to originate from phonological-processing deficits in the sublexical route \citep{Gvion2010, Gvion2012}. Throughout this thesis we explored the rich phenomenology of reading from a computational-modelling, data-driven, perspective. 

Alongside the above evidences, a similarly large body of studies subsists in the literature, supporting an opposite view, which describes dyslexia as resulting from a single cause \citep{stanovich1988explaining, s98, s00, ss05, rrddcw03, rs08, d09, bdvsgmg13, vgpsh13, grggvfb02, r14}. These two opposing views were brought together under the term \textit{The dyslexia debate} \citep{eg14}. The first chapter of this thesis provides the first computational study of reading-error patterns, based on the largest existing corpus of reading errors made by dyslexic people. In particular, it directly addresses the dyslexia debate by exploring reading-error patterns with probabilistic graphical models, spelling out the generation process of reading errors in probabilistic terms. This study is the first to provide support to the subtype approach to dyslexia from a computational-modelling point of view. In addition, we suggest the most predictive model, among those we explored, as an easily accessible and objective diagnosis tool for dyslexia. Although our models were based on error-types, which were manually marked by clinicians, this stage can be automated as well, thus achieving a completely automated screening test. Speech-recognition algorithms have made significant improvement in recent years, and can be efficiently modified for the task of identifying error types such as letter substitution, omission or regularization errors, in both word and nonword reading. With continuously growing databases of reading errors, future research can extend this new line of research, addressing the issue with even larger-scale experiments. Such future study can further benefit both the theoretical and clinical aspects of the field. 

Recent studies have identified a new family of dyslexia subtypes, which involves letter-to-phoneme conversion errors when the target phoneme is composed of specific phonological features such as voicing or nasality \cite{Gvion2010, Gvion2012, kf11}. Motivated by these findings, chapter 2 addresses the question of phoneme similarity from a new computational-modelling perspective. The methodological approach presented in this study enables the quantification of the contribution of each subphonemic feature to perceptual phoneme similarities, in a data-driven manner. Results reveal an order among subphonemic features in English with respect to their perceptual saliency - voicing, nasality, distributed-stridents and approximants. Interestingly, these four leading phoneme groups correspond to observed substitutions errors in reading between phonemes differing in one of the features: voicing, nasality, strident or approximant. Before comparing these, we suggest to first distinguish between substitution errors that can be termed \textit{inter-class} errors, in which a phoneme from one natural class, e.g. [+voice], is substituted with a phoneme from the opposite class [-voice], as when reading {\it this} as 'thiz'. Another example of inter-class error is when reading {\it near} as 'dear', with respect to nasality. Inter-class errors were reported in dyslegzia with respect to voicing, and in nasalexia with respect to nasality. In contrast, we call \textit{intra-class} errors, errors in which a phoneme is substituted with another phoneme from the same class, as when reading {\it sip} as 'ship', with respect to stridency. Intra-class errors were observed for stridents and approximants \citep[unpublished]{friedmann}. In contrast to intra-class errors, voicing and nasality inter-class errors in English seem in variance to the relatively high saliency of these features. One would expect a low rate of inter-class errors for features that are highly discriminatives, since they render the corresponding pair of phonemes less similar and thus less prone to errors. One explanation for this may be that feature-based reading aloud errors do not follow perceptual phoneme similarity, but are also influenced by production processes as well - possible articulatory similarities among phonemes can therefore affect error rates in addition to perceptual similarities\footnote{Note that feature-based reading errors were localized to the sublexical route, before the phonological output buffer and articularoy system, and are therefore different than mere speech errors \citep{Gvion2010}.}. Pairs of phonemes that are relatively perceptually distinct, such as /n/-/d/, can be articulatorily similar (velum lowering in this case). The role of perceptual and articulatory similarities among phonemes in sublexcial reading, silent or reading aloud, is yet unclear and requires future research. Taken together, the discrepancy between perceptual saliency and error prevalence points to the conclusion that perceptual similarities may not solely account for feature-based reading errors in dyslexia. With that said, we would like to point out a relationship between perception and production, proposed by the quantal theory of speech perception which may shed additional light on the question in hand. This theory suggests that languages prefer distinctive features that are optimal in maximizing acoustic dissimilarities while minimizing articulatory effort \citep{stevens1989quantal, stevens2002toward}. Voicing and nasality are therefore optimal in this sense. They have highly discriminative power, as our study shows for English, with relatively minimal articulatory effort (chord vibration and velum lowering, respectively). These features therefore seem liable targets for selective deficits in many languages. Similarly, chapter 2 shows that distributed stridents have relatively high perceptually discriminative power, and have a minimal articulatory difference from non-distributed stridents (alveolar to post-alveolar articulator change). Accordingly, substitution errors such as /g/-/k/ (voicing), /n/-/d/(nasality) and /s/-/\textipa{S}/(distributed-stridency) may be expected to be more frequent and observable, taking into account both perceptual and articulatory similarities.

Chapter 3 further inquires into perceptual phoneme similarity by recording spiking activity from single neurons in the superior temporal gyrus during a phoneme-listening task. Consistently with previous ECoG and EEG studies, our results support a functional organization of phonemes that is dominated by manner-of-articulation features, as revealed also at the cellular level. We found that neural representations of phonemes cluster according to sonorant and obstruent features, and that manner-of-articulation features are better decoded from spiking-activity data compared to place-of-articulation. These results provide additional support to auditory theories of speech perception \citep{stevens1989quantal, stevens2002toward}, and are in contrast to motor theories \citep{liberman1985motor, browman1992articulatory}. Remarkably, spiking activity from a relatively small number of neurons - fourteen, collected from six patients - reflects phoneme similarities derived from behavioral results, based on phoneme-confusion experiments using the same set of stimuli. Specifically, the nasal and approximant features show distinct neural representations compared to other feature classes, which corresponds to their relatively high perceptual saliency as descirbed in chapter 2. 

Overall, the thesis demonstrates the contribution of data-driven computational modelling to the field of reading disorders and neurolinguistics. Specifically, its main contributions are:


\begin{itemize}
   \item  Study 1
   \begin{itemize}
        \item  Support to the subtype approach to dyslexia from a probabilistic point of view.
        \item Laying the ground for an automatization of the diagnosis process of dyslexia, based on error types.
    \end{itemize}
   
    \item  Study 2
    \begin{itemize}
        \item A new methodological approach to the problem of phoneme similarity.
        \item The first phoneme-confusion dataset for Hebrew.
        \item A quantification of the contribution of phonological features to phoneme similarity.
   \end{itemize}
   
   \item Study 3
   \begin{itemize}
        \item A new dataset of single-cell activity in response to phonemic stimuli.
        \item Characterization of the functional organization of phonemes at the cellular level, and a comparison to the cognitive one as derived from behavioral data.
        \item Support to auditory theories of speech perception.
   \end{itemize}
\end{itemize}