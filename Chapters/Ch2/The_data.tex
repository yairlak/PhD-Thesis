\section{The data}
Confusion errors have traditionally been used as a measure for perceptual phoneme similarity, based on the idea that the more confusable two phonemes are the more similar they are. Three phoneme-confusion datasets are explored in this study: (1) the dataset from \citet{NicelyMiller1955}, (2) The dataset from \citet{Luce1987}, and (3) results from a new dataset collected from the experiment in Hebrew.

\subsection{The Nicely and Miller dataset} The classic work by \citet{NicelyMiller1955} analyzed auditory confusion between pairs of 16 consonant phonemes of English with added white noise at different signal-to-noise ratios (SNRs). The noise-corrupted phonemes were presented to subjects in a classification-task paradigm, in a consonant-vowel (CV) form, and the effect of low- and high-pass filtering on phoneme confusion was explored. The resulting confusion matrices from the experiments are provided in their paper. In what follows, we refer to this dataset as the \textit{N\&M dataset}, focusing on their -12dB SNR confusion matrix. Using tools from information theory, N\&M analysed the confusion matrices to see how subphonemic features - voicing, nasality, affrication, duration and place of articulation - affect phoneme confusion, describing subphonemic features as separated information-transfer channels. Their results suggest that the voicing and nasality features are less affected by random noise than the other three features tested: affrication, which distinguishes between /f\textipa{T}s\textipa{S}v\textipa{D}z\textipa{Z}/ and /ptkbdgmn/; duration, which distinguishes between /s\textipa{S}z\textipa{Z}/ and the other phonemes; and place of articulation with three-valued classification - front, middle and back. Affrication and duration were found to be more affected by random noise, but less than the place of articulation feature, which was the most affected by random noise. 

\subsection{The Luce dataset} The work carried by \citet{Luce1987} extended the work by N\&M in several aspects. First, the segment inventory was extended from 16 phonemes to 24 phonemes (see Table A.1 for a full list of phonemes). Second, it used words instead of CV syllables, which is a more natural way of presenting stimuli. Words in the experiment only differed by the initial or the final consonant. Auditory confusion was analyzed for these two cases - initial and final - separately. Similarly to N\&M, auditory confusion was tested at different levels of SNR, using white noise. In what follows, we refer to this dataset as the \textit{Luce dataset}. To have higher confusion rate, we focused on initial-position confusions at the highest available noise level \citep{Redford1999}.

\subsection{A new Hebrew dataset}
To evaluate the perceptual confusability of Hebrew phonemes, we collected data from native Hebrew speakers.

\paragraph{Participants}
Thirty-two native, monolingual, Hebrew speaking participants (13 males, 19 females), ages: 21-35 (mean - 27.1), participated for course credit.

\paragraph{Stimuli}
All stimuli were recorded in an anechoic chamber with a RØDE NT2-A microphone and a Metric Halo MIO2882 audio interface, at a sampling rate of 44.1kHz. Stimuli were generated by two male native speakers of Hebrew. The total number of stimuli was 38 (19 phonemes * 2 speakers). Length and pitch (by semi-tone intervals) were compared across recorded tokens to choose the most highly comparable stimulus-types. This was done by looking at differences in timeline arrangement, using built-in pitch tracker in a commercial software (\textit{Logic Pro X}). Further cleaning of noise residues in high resolution mode was done, applying extremely mild reduction levels (using \textit{Waves X-Noise} software). Next, all stimuli were added white noise at -12 SNR and normalized to the same RMS using a commercial software package (MATLAB R2013.a, \textit{The MathWorks Inc., Natick, MA, 2000}). See table 2.1 for the full list of phonemes.\footnote{Stimuli are available at: "https://github.com/yairlak/The-perceptually-discriminative-power-of-subphonemic-features"; see README file}

\paragraph{Paradigm}
Stimuli were presented to participants in triplets, in an AXB manner, in which each token was a CV syllable. The vowel in all syllables was always /a/ since while front vowels (/i/ and /e/) may trigger fronting or palatalization, and non-low back vowels (/u/ and /o/) cause lip rounding, the low vowel /a/ has little effect on the adjacent consonantal phonemes. At each trial, participants were asked to judge whether the middle phoneme (X) is the same as the first phoneme (A) or the last phoneme (B) by pressing one of two keys on the computer keyboard.

For each pair of phonemes {A, B}, all four possible triplets were presented to the participants: (1) AAB, (2) BAA, (3) ABB, and (4) BBA. If the participant was mistaken on either (1) or (2), the result was marked as a confusion of phoneme A with B. If the participant was mistaken on either (3) or (4), the result was marked as a confusion of the phoneme B with A. 

Sessions were conducted in a quiet room. The stimuli were presented in random order via earphones (Sennheiser HD 280 PRO). The Graphical User Interface (GUI) of the experiment was created with MATLAB, and ran on a laptop (Lenovo Yoga 2 Pro). A training session with 10 trials preceded the experiment. During the experiment, a pause was offered by the GUI to the participant every 20 trials. Participants could then decide to continue by pressing a key on the keyboard.

Outliers were discarded based on response time (RT) as follows. For each subject, the mean RT and standard deviation (SD) were calculated across all trials. Trials with RT above the mean plus three SDs were discarded.

Table 2.2 provides the resulting confusion matrix for all phonemes. The set of phonemes included 19 phonemes (see Table 2.1 for the full list). In what follows, we refer to this dataset as the \textit{Hebrew dataset}.

\begin{landscape}
\renewcommand{\arraystretch}{0.8}
\begin{table}[H]
\centering
 \begin{tabular}{|c||c|c|c|c|c|c|c|c|c|c|c|c|c|c|c|c|c|c|c||c|}
\hline

Ph  & \textipa{b}	&	\textipa{g}	&	\textipa{d}	&	\textipa{h}	&	\textipa{v}	&	\textipa{z}	&	\textipa{X}	&	\textipa{t}	&	\textipa{j}	&	\textipa{k}	&	\textipa{l}	&	\textipa{m}	&	\textipa{n}	&	\textipa{s}	&	\textipa{f}	&	\textipa{p}	&	\textipa{ts}	&	\textipa{K}	&	\textipa{S} & Total \\
\hline
\hline
b	&	282	&	8	&	4	&	6	&	14	&	10	&	15	&	10	&	4	&	14	&	3	&	4	&	4	&	11	&	9	&	5	&	10	&	4	&	18	&	435 \\
g	&	8	&	279	&	5	&	2	&	8	&	6	&	6	&	12	&	4	&	5	&	6	&	2	&	4	&	5	&	6	&	4	&	0	&	5	&	18	&	385 \\
d	&	10	&	12	&	213	&	7	&	7	&	8	&	10	&	4	&	3	&	15	&	3	&	0	&	9	&	7	&	10	&	6	&	12	&	1	&	9	&	346 \\
h	&	8	&	2	&	3	&	385	&	5	&	4	&	9	&	7	&	1	&	11	&	0	&	19	&	9	&	1	&	7	&	9	&	7	&	15	&	8	&	510 \\
v	&	5	&	2	&	1	&	6	&	215	&	7	&	2	&	10	&	5	&	3	&	4	&	5	&	1	&	10	&	3	&	0	&	5	&	9	&	10	&	303 \\
z	&	7	&	8	&	11	&	4	&	6	&	250	&	3	&	8	&	3	&	4	&	5	&	8	&	5	&	12	&	2	&	0	&	5	&	9	&	3	&	353 \\
\textipa{X}	&	13	&	7	&	7	&	5	&	12	&	5	&	323	&	4	&	2	&	6	&	3	&	6	&	7	&	2	&	2	&	6	&	7	&	2	&	1	&	420 \\
t	&	2	&	6	&	17	&	0	&	2	&	4	&	3	&	310	&	2	&	0	&	6	&	0	&	0	&	6	&	1	&	6	&	10	&	5	&	6	&	386 \\
j	&	1	&	6	&	6	&	1	&	4	&	6	&	1	&	1	&	380	&	1	&	8	&	2	&	4	&	3	&	0	&	0	&	1	&	1	&	4	&	430 \\
k	&	7	&	12	&	0	&	4	&	28	&	9	&	3	&	15	&	8	&	292	&	4	&	1	&	8	&	16	&	10	&	5	&	13	&	4	&	13	&	452 \\
l	&	4	&	1	&	4	&	3	&	1	&	5	&	5	&	2	&	3	&	1	&	414	&	5	&	1	&	5	&	0	&	1	&	9	&	2	&	3	&	469 \\
m	&	3	&	2	&	5	&	10	&	4	&	0	&	7	&	1	&	2	&	3	&	1	&	307	&	2	&	3	&	6	&	0	&	1	&	3	&	2	&	362 \\
n	&	2	&	2	&	1	&	0	&	1	&	6	&	3	&	3	&	2	&	1	&	1	&	11	&	279	&	0	&	1	&	0	&	2	&	10	&	2	&	327 \\
s	&	8	&	25	&	10	&	4	&	5	&	3	&	6	&	7	&	7	&	6	&	3	&	4	&	6	&	301	&	8	&	6	&	20	&	6	&	17	&	452 \\
f	&	5	&	4	&	3	&	1	&	8	&	7	&	4	&	11	&	5	&	8	&	1	&	3	&	5	&	6	&	309	&	11	&	3	&	2	&	5	&	401 \\
p	&	7	&	4	&	1	&	5	&	10	&	3	&	12	&	2	&	2	&	7	&	1	&	14	&	10	&	5	&	5	&	370	&	2	&	2	&	1	&	463 \\
ts	&	8	&	30	&	8	&	2	&	11	&	1	&	5	&	9	&	10	&	1	&	2	&	2	&	6	&	0	&	9	&	7	&	343	&	5	&	10	&	469 \\
\textipa{K}	&	1	&	7	&	3	&	0	&	9	&	0	&	1	&	1	&	5	&	6	&	5	&	8	&	2	&	2	&	4	&	9	&	1	&	389	&	1	&	454 \\
\textipa{S}	&	9	&	9	&	23	&	5	&	7	&	10	&	1	&	7	&	5	&	7	&	7	&	4	&	8	&	8	&	4	&	5	&	11	&	6	&	278	&	414 \\
\hline
\hline
Total	&	390	&	426	&	325	&	450	&	357	&	344	&	419	&	424	&	453	&	391	&	477	&	405	&	370	&	403	&	396	&	450	&	462	&	480	&	409	&	7831 \\
 \hline
 \end{tabular}
\caption{Phoneme confusion in Hebrew between 19 phonemes (SNR -12, AXB paradigm). Each value in the matrix represents the confusion $C(i,j)$, which is the number of times that a phoneme in row $i$ was perceived as the phoneme in column $j$.}
\end{table}
\end{landscape}