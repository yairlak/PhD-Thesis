\section{Methods}
We describe learning a metric given a feature set and a phoneme-similarity dataset. Section 2.2.1 describes phoneme representations in feature space. Section 2.2.2 describes different methods for learning optimal mappings. Finally, Section 2.2.3 describes phoneme-confusion datasets that are used in this study, and the way perceptual distances were estimated.

\subsection{Representing phonemes in feature space}
We study phoneme representations in the feature space based on: (a) Articulatory features, and (b) Phonological features. See table 2.1 for the full list of phonemes.

\paragraph{Articulatory features} The first representation scheme uses 14 binary features, based on articulatory features: six place features - labial, dental, alveolar, palatal, velar and glottal, ordered according to the place along the vocal tract (Figure 1B); Seven manner features - plosive, fricative, affricate, lateral, rhotic, nasal, glide, ordered by sonority; And a single voicing feature. As an example, the labial-plosive-unvoiced /p/ is represented  as (1, 0, 0, 0, 0, 0, 0, 1, 0, 0, 0, 0, 0, 0), where the first six components are for place of articulation, the next seven components are for manner and the last vector component is for voicing. Another example, the alveolar-nasal-voiced /n/ is represented as (0, 0, 0, 0, 1, 0, 0, 0, 0, 0, 0, 1, 0, 1).

\paragraph{Phonological features} Phonemes are represented along 12 binary phonological features: consonantal, continuant, strident, voicing, nasality, dorsal, anterior, labial, coronal, distributed and lateral. Continuing with the above examples, /p/ is represented here as (1, 0, 0, 0, 0, 0, 0, 0, 1, 0, 0, 0); and /n/ is represented as (1, 0, 0, 0, 1, 1, 0, 1, 0, 1, 0, 0). \mbox{} \\

Both representation schemes include the minimal set of features that is required to distinguish among all phonemes in the English and Hebrew datasets. Adding more features introduces redundancy among the features. For example, to distinguish between the sets of phonemes in the datasets, the sonority feature is not required in addition to the above list of features. Adding it would introduce redundancy with, e.g., the voicing feature, which might reduce the predictive power of the metric function and its interpretability.


\begin{landscape}% Landscape page

\begin{table}[H]
\centering % Center table
\tiny
\begin{tabular}{|c|c|c|c|c|c|c|c|c|c|c|c|c||c|c|c|c|c|c|c|c|c|c|c|c|c|c|}
\hline

&\multicolumn{12}{|c|}{Phonological features}&\multicolumn{14}{|c|}{Articulatory features}\\
\hline
&	cn	&	ct	&	DR	&	st	&	vc	&	ns	&	dr	&	an	&	lb	&	cr	&	ds	&	lt	&	lb	&	dn	&	al	&	pa	&	vl	&	gl	&	pl	&	af	&	fr	&	ns	&	lt	&	rt	&	gd	&	vc	\\
\hline

${p}^{\star\dagger\ddagger}$	&	+	&	-	&	-	&	-	&	-	&	-	&	-	&	-	&	+	&	-	&	-	&	-	&	+	&	-	&	-	&	-	&	-	&	-	&	+	&	-	&	-	&	-	&	-	&	-	&	-	&	-	\\
$b^{\star\dagger\ddagger}$	&	+	&	-	&	-	&	-	&	+	&	-	&	-	&	-	&	+	&	-	&	-	&	-	&	+	&	-	&	-	&	-	&	-	&	-	&	+	&	-	&	-	&	-	&	-	&	-	&	-	&	+	\\
$t^{\star\dagger\ddagger}$	&	+	&	-	&	-	&	-	&	-	&	-	&	-	&	+	&	-	&	+	&	-	&	-	&	-	&	-	&	+	&	-	&	-	&	-	&	+	&	-	&	-	&	-	&	-	&	-	&	-	&	-	\\
$d^{\star\dagger\ddagger}$	&	+	&	-	&	-	&	-	&	+	&	-	&	-	&	+	&	-	&	+	&	-	&	-	&	-	&	-	&	+	&	-	&	-	&	-	&	+	&	-	&	-	&	-	&	-	&	-	&	-	&	+	\\
$k^{\star\dagger\ddagger}$	&	+	&	-	&	-	&	-	&	-	&	-	&	+	&	-	&	-	&	-	&	-	&	-	&	-	&	-	&	-	&	-	&	-	&	-	&	+	&	-	&	-	&	-	&	-	&	-	&	-	&	-	\\
$g^{\star\dagger\ddagger}$	&	+	&	-	&	-	&	-	&	+	&	-	&	+	&	-	&	-	&	-	&	-	&	-	&	-	&	-	&	-	&	-	&	-	&	-	&	+	&	-	&	-	&	-	&	-	&	-	&	-	&	+	\\
\textipa{tS}$^{\star}$	&	+	&	-	&	+	&	+	&	-	&	-	&	-	&	-	&	-	&	+	&	+	&	-	&	-	&	-	&	-	&	+	&	-	&	-	&	-	&	+	&	-	&	-	&	-	&	-	&	-	&	-	\\
\textipa{dZ}$^{\star}$	&	+	&	-	&	+	&	+	&	+	&	-	&	-	&	-	&	-	&	+	&	+	&	-	&	-	&	-	&	-	&	+	&	-	&	-	&	-	&	+	&	-	&	-	&	-	&	-	&	-	&	+	\\
$f^{\star\dagger\ddagger}$	&	+	&	+	&	-	&	-	&	-	&	-	&	-	&	-	&	+	&	-	&	-	&	-	&	+	&	-	&	-	&	-	&	-	&	-	&	-	&	-	&	+	&	-	&	-	&	-	&	-	&	-	\\
$v^{\star\dagger\ddagger}$	&	+	&	+	&	-	&	-	&	+	&	-	&	-	&	-	&	+	&	-	&	-	&	-	&	+	&	-	&	-	&	-	&	-	&	-	&	-	&	-	&	+	&	-	&	-	&	-	&	-	&	+	\\
\textipa{T}$^{\star\dagger}$	&	+	&	+	&	-	&	-	&	-	&	-	&	-	&	+	&	-	&	+	&	-	&	-	&	-	&	+	&	-	&	-	&	-	&	-	&	-	&	-	&	+	&	-	&	-	&	-	&	-	&	-	\\
\textipa{D}$^{\star\dagger}$	&	+	&	+	&	-	&	-	&	+	&	-	&	-	&	+	&	-	&	+	&	-	&	-	&	-	&	+	&	-	&	-	&	-	&	-	&	-	&	-	&	+	&	-	&	-	&	-	&	-	&	+	\\
$s^{\star\dagger\ddagger}$	&	+	&	+	&	-	&	+	&	-	&	-	&	-	&	+	&	-	&	+	&	-	&	-	&	-	&	-	&	+	&	-	&	-	&	-	&	-	&	-	&	+	&	-	&	-	&	-	&	-	&	-	\\
$z^{\star\dagger\ddagger}$	&	+	&	+	&	-	&	+	&	+	&	-	&	-	&	+	&	-	&	+	&	-	&	-	&	-	&	-	&	+	&	-	&	-	&	-	&	-	&	-	&	+	&	-	&	-	&	-	&	-	&	+	\\
\textipa{S}$^{\star\dagger\ddagger}$	&	+	&	+	&	-	&	+	&	-	&	-	&	-	&	-	&	-	&	+	&	+	&	-	&	-	&	-	&	-	&	+	&	-	&	-	&	-	&	-	&	+	&	-	&	-	&	-	&	-	&	-	\\
\textipa{Z}$^{\dagger}$	&	+	&	+	&	-	&	+	&	+	&	-	&	-	&	-	&	-	&	+	&	+	&	-	&	-	&	-	&	-	&	+	&	-	&	-	&	-	&	-	&	+	&	-	&	-	&	-	&	-	&	+	\\
$h^{\star\dagger\ddagger}$	&	+	&	+	&	-	&	-	&	-	&	-	&	-	&	-	&	-	&	-	&	-	&	-	&	-	&	-	&	-	&	-	&	-	&	+	&	-	&	-	&	+	&	-	&	-	&	-	&	-	&	-	\\
$m^{\star\dagger\ddagger}$	&	+	&	-	&	-	&	-	&	+	&	+	&	-	&	-	&	+	&	-	&	-	&	-	&	+	&	-	&	-	&	-	&	-	&	-	&	-	&	-	&	-	&	+	&	-	&	-	&	-	&	+	\\
$n^{\star\dagger\ddagger}$	&	+	&	-	&	-	&	-	&	+	&	+	&	-	&	+	&	-	&	+	&	-	&	-	&	-	&	-	&	+	&	-	&	-	&	-	&	-	&	-	&	-	&	+	&	-	&	-	&	-	&	+	\\
\textipa{N}$^{\star}$	&	+	&	-	&	-	&	-	&	+	&	+	&	+	&	-	&	-	&	-	&	-	&	-	&	-	&	-	&	-	&	-	&	+	&	-	&	-	&	-	&	-	&	+	&	-	&	-	&	-	&	+	\\
$l^{\star\ddagger}$	&	+	&	+	&	-	&	-	&	+	&	-	&	-	&	+	&	-	&	+	&	-	&	+	&	-	&	-	&	+	&	-	&	-	&	-	&	-	&	-	&	-	&	-	&	+	&	-	&	-	&	+	\\
\textipa{R}$^{\star}$	&	+	&	+	&	-	&	-	&	+	&	-	&	-	&	-	&	-	&	+	&	-	&	-	&	-	&	-	&	+	&	-	&	-	&	-	&	-	&	-	&	-	&	-	&	-	&	+	&	-	&	+	\\
$w^{\star}$	&	-	&	+	&	-	&	-	&	+	&	-	&	+	&	-	&	+	&	-	&	-	&	-	&	+	&	-	&	-	&	-	&	-	&	-	&	-	&	-	&	-	&	-	&	-	&	-	&	+	&	+	\\
$j^{\star\ddagger}$	&	-	&	+	&	-	&	-	&	+	&	-	&	-	&	-	&	-	&	+	&	+	&	-	&	-	&	-	&	-	&	+	&	-	&	-	&	-	&	-	&	-	&	-	&	-	&	-	&	+	&	+	\\
\textipa{X}$^{\ddagger}$	&	+	&	+	&	-	&	-	&	-	&	-	&	+	&	-	&	-	&	-	&	-	&	-	&	-	&	-	&	-	&	-	&	+	&	-	&	-	&	-	&	+	&	-	&	-	&	-	&	-	&	-	\\
\textipa{ts}$^{\ddagger}$	&	+	&	-	&	+	&	+	&	-	&	-	&	-	&	+	&	-	&	+	&	-	&	-	&	-	&	-	&	+	&	-	&	-	&	-	&	-	&	+	&	-	&	-	&	-	&	-	&	-	&	-	\\
\textipa{K}$^{\ddagger}$	&	+	&	+	&	-	&	-	&	+	&	-	&	+	&	-	&	-	&	-	&	-	&	-	&	-	&	-	&	-	&	-	&	+	&	-	&	-	&	-	&	-	&	-	&	-	&	+	&	-	&	+	\\

\hline

\end{tabular}

\caption{Phonological and articulatory features of all phonemes in the datasets. cn=consonantal; ct=continuant; DR=delayed release; st=strident; vc=voicing; ns=nasal; dr=dorsal; an=anterior; lb=labial; cr=coronal; ds=distributed; lt=lateral; dn=dental; al=alveolar; pa=postalveolar; vl=velar; gl=glottal; pl=plosive; af=affricate; fr=fricative; lt=lateral; rt=rhotic; gd=glide. $^{\star}$Luce, $^{\dagger}$N\&M, $^{\ddagger}$Hebrew dataset}

\end{table}
\end{landscape}

\subsection{Metric learning}
We explore two approaches to learn a metric function \citep{Kulis2012}: (1) viewing the problem as a least-squares problem and (2) viewing the problem as a large-margin ranking problem \citep{Chechik2010}. We begin by providing a formal definition of the problem, describing the terms, variables and parameters of the proposed model.

Let $A$ be a set, a metric function $d: A \times A \to \mathbb{R}$ is a non-negative symmetric function over pairs of elements of $A$, which assigns a value (a distance) to every pair in the set, and satisfies the triangle inequality and that for every element $x$ in $A$: $d(x, x) = 0$.

In our case, the elements of the set are phonemes $\{\bar{p}^1, ..., \bar{p}^{n_p}\}$, where $n_p$ is the number of phonemes. We describe each phoneme as a vector of size $n_f$, $\bar{p}^i \in \mathbb{R} ^{n_f}$, and define the following metric $d_{ij}(\bar{p}^i, \bar{p}^j)$ over pairs of phonemes:

\begin{equation}
    d_{ij}(\bar{p}^i, \bar{p}^j) = (\bar{p}^i - \bar{p}^j)^T W (\bar{p}^i - \bar{p}^j),
\end{equation}

where $W \in \mathbb{R} ^{n_f \times n_f}$ is a positive semi-definite (PSD) real-valued matrix. This function is a proper metric since it satisfies all conditions: non-negativity (PSD), symmetry, the triangle inequality, and that the distance between a phoneme to itself is always zero. 

Geometrically, this metric can be described as: (1) performing a linear mapping on a given pair of vectors from the feature space; and (2) calculating the square Euclidean distance between the results. To see this, note that a PSD matrix $W$ can be decomposed such that $W = \widetilde{W}^T\widetilde{W}$. Given a pair of phoneme vectors, $\bar{p}^i$ and $\bar{p}^i$, the mapped vectors are $\widetilde{W}\bar{p}^i$ and $\widetilde{W}\bar{p}^j$, and the square Euclidean distance between a pair of mapped phonemes is therefore: $||\widetilde{W}\bar{p}^i - \widetilde{W}\bar{p}^j||^2 = (\bar{p}^i - \bar{p}^j)^T\widetilde{W}^T\widetilde{W}(\bar{p}^i - \bar{p}^j)$, which is the right-hand side of Eq(2.1).

The free parameters of the model are the elements of $W$. Given a set of perceived distances ${D_{ij}}$ between phonemes, as measured in an experiment, an optimal set of model parameters $W^\ast$ would bring the model distances $d_{ij}$ as close as possible to the empirical distances $D_{ij}$. Formally, we define the optimization problem:

\begin{equation}
    W^\ast = \min_{W \in S^n_+}{{\sum_{{i} < {j}}{(d_{ij} - D_{ij})^2}}},
\end{equation}

where the sum is over all ${n_p \choose 2}$ pairs of phonemes, and $S^n_+$ denotes the set of $n \times n$ PSD matrices. We describe below two methods to solve the optimization problem. The first reduces the problem into a least-squares (LS) problem, the second reduces it into a large-margin ranking problem (OASIS, \citealp{Chechik2010}). We also test a variant of each of the methods, by adding a diagonal constraint on the weight matrix (section 2.2.2.3), thus having four methods in total. We refer to these four ways of learning a metric function: LS, diagonal-LS, OASIS, diagonal-OASIS.

\subsubsection{Method A: a least-squares formulation}
Formulating the problem as a least-squares problem will allow the use of standard tools to solve the problem. We now show how the problem of metric learning as defined in section 2.2.2, can be described as a least-squares problem:

We define a loss function $\mathcal{L}$ that correspond to the optimization problem in Eq(2.2).

\begin{equation}
    \mathcal{L} = \sum_{{i} < {j}}{((\bar{p}^i - \bar{p}^j)^TW(\bar{p}^i - \bar{p}^j) - D_{ij})} + \lambda||w||_1^2 \quad ,
\end{equation}

where L1-regularization is added to the loss function in its second term. Adding L1-regularization has two advantages. First, it mitigates overfitting to the empirical training data. Second, L1-regularization results in sparser solutions, i.e., a larger number of parameters have values equal to zero \citep{Tibshirani1996}. Importantly, in our case, where feature weights are derived from data collected from a perceptual-discrimination task, non-zero weights that endured this penalty are those that are necessary for explaining the measured perceptual distances. Features that have higher weight are therefore more perceptually discriminative. 

Rewriting the first term in the objective function, we get a classic LS formulation in the following matrix notation: $||\widetilde{P}\bar{w} - D||_2^2 + \lambda||w||_1^2$, where $\bar{w}$ is an $n_f^2$-dimensional vector, and $\widetilde{P}$ is an ${n_p \choose 2} \times n_f^2$ matrix. A row of $\widetilde{P}$ corresponds to the distance between the $i^{th}$ and $j^{th}$ phonemes, and a column corresponds to the weight of the $k^{th}$ and $l^{th}$ features from the quadratic form in Eq(2.1). The elements of $\widetilde{P}$ are therefore of the form: $(p_k^i-p_k^j)(p_l^i-p_l^j)$.

To solve the L1-regularized least-squares problem, we use the LASSO algorithm \citep{Tibshirani1996}. The regularizer size $\lambda$ is determined by comparing predictions of models with different values of $\lambda$'s. Values of $\lambda$ were examined for 18 values in the range $[10^{-3}, 10^5]$. The optimal size of $\lambda$ was determined according to the maximal prediction on a validation set. We refer to this method of learning a metric function as LS.

\subsubsection{Method B: a large-margin ranking problem} The goal here is to learn a metric function $d(p^i, p^j)$ as in Eq(2.1), that obeys the following condition \citep{Chechik2010}: a pair of phonemes ($p^i, p^i_+$) that are perceptually closer than another pair ($p^i, p^i_-$), are also closer according to the learned metric function. Formally, for all triplets of phonemes:

\begin{equation}
\forall{p^i, p^i_+, p^i_-}: D(p^i, p^i_+)<D(p^i, p^i_-) \rightarrow d(p^i, p^i_+)<d(p^i, p^i_-)
\end{equation}

The hinge-loss function to the constraint in Eq(2.4) together with a margin is:
\begin{equation}
    l_W(p^i, p^i_+, p^i_-) = max\{0, 1 + d(p^i, p^i_+) - d(p^i, p^i_-)\}.
\end{equation}

The optimization problem and the algorithm are then solved as in \citet{Chechik2010}. We refer to this method as OASIS.

Note that in the LS method, the exact values from the empirical data are used to learn the parameters of the model $W$, whereas in OASIS, the data values are used for ranking the empirical distances between phonemes. It is an empirical matter which of the two methods better fits our problem, we therefore explore metric learning of phonemes with both LS and OASIS.

\subsubsection{Diagonal constraint} In addition, we test a version of the above two methods when a diagonal-constraint on the parameter matrix $W$ is added. That is, the off-diagonal weights in $W$ are set to zero. The advantage of adding a diagonal constraint is that the model has fewer parameters and it is thus less prone to overfitting. On the other hand, too simple a model may not be able to capture the full structure of the data. We test each method with and without a diagonal constraint. 

\subsection{Baseline methods}
We compare the predictions of the models to three baselines: \textit{uniform weights}, \textit{PMV} \citep{ladefoged2014course} and \textit{Frisch similarity \citep{Frisch1997}}.

\paragraph{Baseline \#1: Uniform weights} $W$ is the identity matrix, with all weights set to 1 on the diagonal, and all other weights to zero.

\paragraph{Baseline \#2: Place-Manner-Voicing (PMV)} A common measure of phoneme similarity counts the number of articulatory dimensions - place, manner and voicing - that are shared by two phonemes. Formally,

\begin{equation}
S_{PMV}(p_i, p_j) = \delta_{place(p_i),place(p_j)} + \delta_{manner(p_i),manner(p_j)} + \delta_{voicing(p_i),voicing(p_j)} \quad ,
\end{equation}

where $\delta_{i,j}$ is the Kronecker delta.

\paragraph{Baseline \#3: Frisch similarity \citep{Frisch1997}} Similarity is calculated based on shared and non-shared natural classes, instead of features:
 
\begin{equation}
\resizebox{.9\hsize}{!}{$S_{FRISCH}(p_i, p_j) = \frac{\#(shared \quad natural \quad classes)}{\#(shared \quad natural \quad classes) + \#(non-shared \quad natural \quad classes)}$}.
\end{equation}

By introducing natural classes, similarity becomes dependent on the redundancy of the features. Non-redundant features have more affect on similarity than redundant ones. In particular, a totally redundant feature adds no new natural classes and therefore does not affect similarity. This introduces a weighing to the similarity function that is based on redundancy. \mbox{} \\

\subsection{Perceptual distances from confusion data}
We now describe how the perceived distances ${D_{ij}}$ are estimated from empirical data. First, phoneme similarity is calculated from confusion data following Shepard's method:

\begin{equation}
    S_{ij}=\frac{p_{ij}+p_{ji}}{p_{ii}+p_{jj}},
\end{equation}

where $p_{ij}$ is the proportion of times that $i$ tokens were called $j$. Perceived distances are then calculated following Shepard's law \citep{Shepard1987, Johnson2004}:

\begin{equation}
    D(p^i, p^j) = -ln(S(p^i, p^j)).
\end{equation}

This assumes that the relationship between perceptual distances and similarity is exponential. \citet{Ennis1988} showed that this law can account for various empirical findings about object confusion (in same-different tasks) if decisions between objects are described in a probabilistic framework. That is, percepts are treated as probabilistic and Shepard's law is applied at the stage of decision between objects.

\subsection{Model evaluation}
To compare the quality of the proposed methods, we evaluate the predictive power of the metric function of each method using a cross-validation procedure. First, for each method, we learn a metric function from a subset of the data. We then assess the prediction of the learned function on left-out data. We repeat this by leaving out a different subset of the data each time, finally averaging the resulting predictions across all left-out subsets. More specifically, we use a leave-one-phoneme-out procedure, where each time, we leave out all perceived distances involving one phoneme. We then repeat this for all phonemes. Finally, we report the average of all resulting values. 

We use the Spearman rank correlation coefficient as a prediction measure. For each left-out phoneme $p$, we calculate the Spearman correlation $\rho_{d^p, D^p}$, between the model-predicted distances to the left-out phoneme $d^p$, and the empirical ones $D^p$. The final evaluation measure of model prediction is the average of this measure across all left-out phonemes $\bar{\rho} = \frac{1}{n_p}\sum_{p=1}^{n_p}\rho_{d^p, D^p}$. According to the distribution of the values $\rho_{d^p, D^p}$ we evaluate statistical significance between different methods.