\begin{abstract}

Dyslexia is a common reading disorder, characterized by high rates of reading errors and slow reading speeds, alongside normal intelligence. Research in the past decades has achieved major understandings about the processes involved in normal and abnormal reading, yet there is not one well accepted theory of dyslexia. Particularly, an ongoing debate remains on whether dyslexia is one central malfunction, with a single cause, or rather a general term describing multiple disorders, termed subtypes of dyslexia. The latter view relies on the rich phenomenology of reading errors described in research, to identify specific patterns of errors, corresponding to different subtypes of the disorder. For example, a dyslexic person may read the word {\it tries} as 'tiers', as 'trial', or as 'dries'. The first error is characteristic to letter-transposition dyslexia, the second to visual dyslexia and the third to voicing-substitution dyslexia (the phonemes /t/ and /d/ differ by the phonological feature of voicing). Following this line of research, the study of subtypes of dyslexia has recently raised more specific questions in the field. In particular, recent studies have identified new subtypes of dyslexia, characterized by mapping a letter to an erroneous phoneme. This phoneme typically differs only by a single phonological feature from the correct one, as in the above example of voicing substitution. These phonological feature-based subtypes of dyslexias were found to affect specific phonological features but not others, yet it remains unclear why only specific features are affected. In this thesis, we introduce a novel approach to the study of these questions, through computational data-driven models, a methodology which has rarely been adopted to explore reading and its disorders. Chapter 1 addresses the debate on the heterogeneity of dyslexia by constructing data-driven probabilistic graphical models to analyze patterns of reading errors made by dyslexic people. The point of departure and motivation of chapters 2 and 3 is the question regarding the uneven prevalence of phonological feature-based subtypes of dyslexia. We address it through studying similarity relations among phonemes, as phonological errors presumably stem from similarity between phonemes. Chapter 2 establishes a new approach to study phoneme similarity, which enables the quantification of the contribution of various phonological features to overall perceptual phoneme similarity, based on behavioral data. Different contributions of phonological features to phoneme similarity may explain the prevalence of specific feature-based dyslexia subtypes. Chapter 3 studies the similarity structure among phonemes, based on data of neural-activity recordings. In this chapter, we present and characterize data of single-cells activity recorded from high-level auditory regions, collected from neurosurgical patients who performed a listening task with phonemic stimuli. Results of this thesis provide support to the subtype approach to dyslexia (chapter 1), and describe the dominance of various phonological features in perceptual phoneme similarities, as revealed from behavioral and neuronal data (chapters 2 and 3). We discuss the prevalence of phonological feature-based subtypes of dyslexia, in view of these results. We demonstrate that insights about persisting questions and debates in the field may be reached from the yet unexplored perspective of computational data driven models.

\end{abstract}