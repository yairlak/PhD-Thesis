\begin{abstract}

Dyslexia is a common reading disorder, characterized by high rates of reading errors, and slow reading speeds, alongside normal intelligence. Research in the past decades has achieved major understandings about the processes involved in normal and abnormal reading, yet there is not one well accepted theory of dyslexia. Particularly, an ongoing debate remains on whether dyslexia is one central malfunction, as single-cause theories argue, or rather a general term describing multiple disorders, as the subtype approach to dyslexia argues. The latter view relies on the rich phenomenology of reading errors identified in research on dyslexia subtypes. For example, A dyslexic person may read the word {\it tries} as 'tiers', as 'dries', or as 'trial'. The first error may be a letter-transposition error, the second may be a voicing-substitution error (since the phonemes /t/ and /d/ differ by voicing), and the third may be a visual error made on the right side of the word - each error corresponding to a different subtype of dyslexia. Research on subtypes of dyslexia has recently required turning to more specific questions in the field: recent studies have identified new subtypes of dyslexia, characterized by mapping a letter to an erroneous phoneme that differs by a single phonological feature compared to the correct one, as in the above example about voicing substitution. These phonological feature-based subtypes of dyslexias were found to involve specific phonological features but not others. It therefore remains unclear why some phonological features are more prone to participate in such substitution errors than others. Introducing a novel approach, this thesis addresses these questions regarding the subtype approach to dyslexia from a new,  computational, perspective. Chapter 1 addresses the debate regarding the heterogeneity of dyslexia by exploring probabilistic graphical models to analyze patterns of reading errors made by dyslexic people. Results support a model assuming multiple dyslexia subtypes, that of a heterogeneous view of dyslexia. Moreover, the models achieve good diagnosis-prediction performance when compared to labels given by clinicians, thus providing the first automated diagnosis tool for dyslexia. The next two chapters address the question regarding the prevalence of phonological feature-based subtypes of dyslexia. Phonological feature-based errors may follow similarity relations among phonemes, assuming the more similar two phonemes are the more prone they are to confusion. Chapter 2 establishes a new approach to study phoneme similarity, which enables the quantification of the contribution of various phonological features to overall perceptual phoneme similarity, based on behavioral data. Different contributions of phonological features to phoneme similarity may explain the prevalence of specific feature-based dyslexia subtypes. Results show that for English, the voicing, nasality, distributed-strident and approximant features have the highest contribution to perceptual distances among phonemes. For Hebrew, a similar order among phonological features emerges, however, the voicing feature was found to be less perceptually discriminative compared to English. In both languages, it is shown that manner-of-articulation features dominate the similarity structure among phonemes. Departing from behavioral data, chapter 3 study the similarity structure among phonemes, based on neuronal data. In this chapter, we present and characterize data of single-cells activity recorded from high-level auditory regions, collected from six neurosurgical patients who performed a listening task with phoneme stimuli. Results show that manner-of-articulation features also dominate the functional organization of phonemes in high-level auditory regions, as revealed by spiking activity, similarly to the results from chapter 2. Directly comparing the similarity structure revealed by spiking activity and that from behavioral results from chapter 2, we find a moderate correlation between the two. Finally, drawing on the results, we discuss the signification of these findings with respect to speech perception - we argue that our results provide support to auditory theories, and are contrast to gestural theories of speech perception. In sum, in the center of this thesis and motivating the questions it addresses is the rich phenomenology of error types in dyslexia. The question of the heterogeneity of dyslexia, and that of the existence of specific feature-based subtypes, are hereby addressed. In common to both investigations, the thesis adopts a novel methodological approach to reading, which designs computational models to study these questions. This methodology has been hardly adopted to explore questions regarding reading and its disorders, this thesis therefore suggests new insights about persisting questions and debates in the field from yet unexplored perspective.

\end{abstract}